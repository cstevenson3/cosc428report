\documentclass{IEEEtran}

\begin{document}

\title{Automated Conversion of Sketches into Source Game Engine Maps}
\author{Cameron Stevenson}
 
\begin{abstract}
abstract...
\end{abstract}

\maketitle

\section{Introduction}

\section{Proposed Method}
\subsection{Segmentation}
We aim to find the segments divided by pencil lines.

The sketch is grayed, inverted, and blurred. An adaptive threshold highlights both sides of each pencil mark as black, with the background in white. This is inverted so pencil edges are white against a black background. A closing operation with 2 dilations and 1 erosion of a 5x5 kernel joins the two sides of each pencil mark, giving thick white pencil against a black background. These white edges may contain black pixels within them. This is acceptable as long as the edge has no black path between segments.

Now that black segments are divided by solid white boundaries, flood fill is used to identify internal segments and the external background. Each pixel in the image which is still black is used as a seed point for flood fill with intensity value 128. A segment is accepted as internal if it is large enough (more than a few pixels) and does not touch the image border. These segments are given intensity value 192 to indicate they are discovered. Rejected segments are given intensity value 255, so that they are treated as equivalent to pencil.

\bibliographystyle{IEEEtran}
\bibliography{references}

\end{document}